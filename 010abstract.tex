\noindent KEYWORDS: \hspace*{0.5em} \parbox[t]{4.4in}{\LaTeX ; Thesis;
  Style files; Format.}

\vspace*{24pt}

\noindent The Knowledge Base (KB) used for real-world applications, such as booking a movie or restaurant reservation, keeps changing over time. End-to-end neural networks trained for these task-oriented dialogs are expected to be immune to any changes in the KB. However, existing approaches breakdown when asked to handle such changes. We propose an encoder-decoder architecture (\sys) with a novel Bag-of-Sequences (\textsc{BoSs}) memory, which facilitates the disentangled learning of the response's language model and its knowledge incorporation. Consequently, the KB can be modified with new knowledge without a drop in interpretability. We find that \sys\ outperforms state-of-the-art models, with considerable improvements (\textgreater10\%) on bAbI OOV test sets and other human-human datasets. We also systematically modify existing datasets to measure disentanglement and show \sys\ to be robust to KB modifications.


Dialog systems or chatbots are computer programs that can interact with humans either using speech interface or text interface. Building dialog systems are gaining popularity due to two major reasons. One to accomplish a task, such as purchasing a mobile phone from Amazon, internet users prefer a simple chat interface compared to navigating through websites or mobile app. Two, mobile phone users spend most of their time using email or messaging applications.

Based on the application, dialogs systems can be divided into two categories: open domain and task oriented. Dialog systems that converse with an intention to accomplish a task such as  recommending a restaurant or booking a flight tickets are task oriented dialog systems. Building task oriented dialog systems requires a considerable effort to define hand crafted features and rules. Research in open domain dialog systems have progressed to a state where given a large corpus of conversation logs, the deep learning models can learn to converse end-to-end without the need of defining hand crafted, domain specific rules. Task oriented dialog systems such  restaurant recommendation system requires the system to consult a knowledge base of restaurants to accomplish the task. Most of the research on modeling dialog systems has been focused on only learning to converse by remembering how conversation are sustained in the training examples. There has been very little work around on how to learn an end-to-end task oriented dialog system that requires access to a knowledge base to accomplish a given task. 

The existing end-to-end task oriented dialog system which uses knowledge base  perform well only on open domain dialog system evaluation metrics, a simple analysis shows that there exists a huge gap when evaluated using task specific metrics. The failure is mostly due to the inability to handle OOV words, inability to perform simple reasoning over knowledge base results such as suggest without repetition and sorting based on a field over.

We first solve the limitations in the existing model by proposing a deep network that can consume knowledge base results and perform basic reasoning. To accomplish this we propose a hierarchical attention network with the ability to perform location based addressing. The overall goal of this research is to learn a usable task oriented dialog system from long human-human chat transcripts. To achieve the goal, we propose to solve the following problems: one, dialog system that can perform complex reasoning such as inferring from more than one knowledge base result to generate a response. The existing systems access the knowledge base just once during the conversation, we propose to extend this by modeling a system that is capable to conversing by accessing the knowledge base more than once. For example, when purchasing a product such as mobile phone, the user describes her requirements,  based on the mobile phones available in the knowledge base, the system should help narrow down the option based on the results. We then wish to work on knowledge bases that contains semi-structured fields along with the structured fields. Finally, we wish to learn a usable dialog system using human to human conversations.